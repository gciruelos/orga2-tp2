\section{HSL}

\subsection{C}
El código de C, al igual que el resto, es bastante sencillo. Lo que hace es loopear sobre todos los píxeles, hacer una conversión de RGB a HSL, hacer las sumas correspondientes, y luego volver a convertir a RGB.

Lo malo de la implementacion es que el código sin optimizar de C hace más operaciones de las necesarias, ya que no usa todo el poder de las operaciones en SSE, que nosotros intentamos utilizar al máximo.

\subsection{ASM1}

En la versión primera versión del código de assembler la operatoriaes bastante distinta a la de C.
Al principio calculamos en xmm4 el vector de números que debemos sumarle a cada pixel hsl, con los parámetros que nos pasaron. De esta manera, 

\xmm{4}
\regfloats{l}{s}{h}{0}

Donde h,s,l son los que nos pasaron como parámetro y el 0 es lo que le tenemos que sumar a la transparencia (nada). Este registro tenemos que guardarlo en la pila, dado que cuando llamamos a rgbTOhsl, nos puede pisar los registros xmm pues la calling convention de C no especifica nada sobre que no se puedan pisar (de hecho en algunos casos lo pisa, fue un bug que tardamos en encontrar).

Tambien tenemos que malloc'ear un float para llamar a las funciones rgbTOhsl y hslTOrgb. Podríamos usar la pila, pero nos resultó mas fácil usar este método.

Luego comenzamos a loopear. 

Lo primero que hacemos es llamar a la función rgbTOhsl, de manera que en nuestro puntero a float que malloc'eamos nos que nos va a quedar el valor hsl del pixel en el puntero.

Lo que hacemos después es recuperar los parámetros, que estaban en la pila (dado que posiblemente el llamado a rgbTOhsl haya pisado el registro).

Luego cargamos los registros que vamos a suar para comparaciones y sumas y restas y finalmente guardamos el valor del pixel en hsl en xmm3

\xmm{3}
\regfloats{LL, SS, HH, AA}

Luego sumamos este registro con el registro que contiene los parámetros, como indica el filtro, de manera que queda

\xmm{3}
\regfloats{l+LL}{s+SS}{h+HH}{AA}

Ahora comienza la operatoria de saturación, entonces vamos a armar los siguientes registros

\xmm{5}
\regfloats{1-(l+LL)}{1-(s+SS)}{-360}{0}

\xmm{6}
\regfloats{-(l+LL)}{-(s+SS)}{360}{0}


Notar que lo que necesitamos hacer esto





