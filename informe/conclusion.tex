\section{Conclusión}

La primera conclusión de este TP es que programar en assembler, aunque díficil, es remunerador. Si se programa usando el paradigma SIMD de buena manera, la performance que se obtiene es realmente alta. Por esta razón creemos que es imporante conocer el paradigma y la tecnología SSE, dado que son la herramienta que nos permite hacer el tuneo más fino de nuestras aplicaciones.

No obstante eso, es importante remarcar programar en assembler con SSE no es una buena alternativa a C para otro tipo de problemas. SSE se destaca en cálculos paralelizables, muy del estilo de los que tuvimos que realizar en el TP.

La segunda conclusión, es que, aunque en el filtro HSL no obtuvimos el desempeño esperado, analizando las razones aprendimos mucho sobre optimizaciones y como funcionan realmente algunos componentes de la PC. Por ejemplo, pudimos notar que realizar shuffles es más rápido que hacer shifts, que al principio parece algo innecesario.

También nos vimos obligados a leer el output de assembler generado por C, para comparar y buscar en la raíz del problema respuestas a nuestras preguntas. Con esto también aprendimos que el compilador de C hace muchas cosas que realmente exceden la capacidad de un programador. Esto está relacionado en parte con lo que dijimos anteriormente de la utilidad de SSE para algunos tipos de problemas. 







